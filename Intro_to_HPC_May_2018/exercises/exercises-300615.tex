% $Header: /Users/joseph/Documents/LaTeX/beamer/solutions/generic-talks/generic-ornate-15min-45min.en.tex,v 90e850259b8b 2007/01/28 20:48:30 tantau $

\documentclass{beamer}

% Based on solutions/generic-ornate-15min-45min.en.tex
% and HPCS-UoC-Beamer-master/uoc-beamer-template-1.tex.

% Copyright 2004 by Till Tantau <tantau@users.sourceforge.net>.
%
% In principle, this file can be redistributed and/or modified under
% the terms of the GNU Public License, version 2.
%
% However, this file is supposed to be a template to be modified
% for your own needs. For this reason, if you use this file as a
% template and not specifically distribute it as part of a another
% package/program, I grant the extra permission to freely copy and
% modify this file as you see fit and even to delete this copyright
% notice. 


\mode<presentation>
{
  \usetheme{cambridge}
  %\usetheme{Warsaw}

  \setbeamertemplate{navigation symbols}{}
  \setbeamercovered{transparent}
  % or whatever (possibly just delete it)
}

\usepackage[english]{babel}
\usepackage[latin1]{inputenc}
\usepackage{multicol}

%\usepackage{times}
%\usepackage[T1]{fontenc}
% Or whatever. Note that the encoding and the font should match. If T1
% does not look nice, try deleting the line with the fontenc.


\title[HPC: An introduction --- Exercises] % (optional, use only with long paper titles)
{An Introduction to High Performance Computing and the HPCS --- Exercises}

%\subtitle
%{Presentation Subtitle} % (optional)

\author[SJ Rankin] % (optional, use only with lots of authors)
{Stuart Rankin\\ \texttt{sjr20@cam.ac.uk}}
%{F.~Author\inst{1} \and S.~Another\inst{2}}
% - Use the \inst{?} command only if the authors have different
%   affiliation.

\institute[HPCS, University of Cambridge] % (optional, but mostly needed)
{High Performance Computing Service (http://www.hpc.cam.ac.uk/)\\
University Information Services (http://www.uis.cam.ac.uk/)}
% - Use the \inst command only if there are several affiliations.
% - Keep it simple, no one is interested in your street address.

\date[01/07/2015] % (optional)
{1st July 2015 / UIS Training}

\subject{Courses --- Exercises}
% This is only inserted into the PDF information catalog. Can be left
% out. 



% If you have a file called "university-logo-filename.xxx", where xxx
% is a graphic format that can be processed by latex or pdflatex,
% resp., then you can add a logo as follows:

% \pgfdeclareimage[height=0.5cm]{university-logo}{university-logo-filename}
% \logo{\pgfuseimage{university-logo}}



% Delete this, if you do not want the table of contents to pop up at
% the beginning of each subsection:
%\AtBeginSubsection[]
%{
%  \begin{frame}<beamer>{Outline}
%    \tableofcontents[currentsection,currentsubsection]
%  \end{frame}
%}


% If you wish to uncover everything in a step-wise fashion, uncomment
% the following command: 

%\beamerdefaultoverlayspecification{<+->}


\begin{document}

\begin{frame}
  \titlepage
\end{frame}

\section{Login}
\begin{frame}{Exercise 1: Login}
\begin{itemize}
\item{Using ssh, login to your HPCS training account.}
\begin{description}
\item[\emph{Hints:}]{\small The remote host is \alert{login.hpc.cam.ac.uk} and the username is the same as for your MCS Desktop training account (i.e. \alert{z4XY}).\hfill\\\smallskip
{\scriptsize N.B.\ If in doubt about the name of your training account, check the number of your station (see the label on the top of the box), then station 1\textbf{XY} should correspond to account z4\textbf {XY}.}\hfill\\\smallskip
First, change directory to the MobaXterm folder, where the private key we are using for authentication lives. The commands are\hfill\\
\alert{cd $\tilde{}$/MobaXterm}\hfill\\
\alert{ssh -Y -i id\_rsa z4XY@login.hpc.cam.ac.uk}\hfill\\
The passphrase when prompted is \alert{\emph{``Introduction to HPC''}}. Note the -Y option which enables transparent forwarding to your X windows display (i.e.\ allows X applications on the cluster to display on your screen).}
\end{description}
\end{itemize}
\end{frame}

\section{File transfer}
\begin{frame}{Exercise 2: File transfer}
\begin{itemize}
\item{rsync the file \alert{exercises.tgz} to your HPCS training account.}
\visible<2->{\begin{description}
\item[\emph{Hints:}]{\small The remote host, username, private key and passphrase are the same as in the previous exercise. Type \alert{cd} first to return to the top directory. The command is\hfill\\
\smallskip
\rightline{\alert{rsync -e 'ssh -i MobaXterm/id\_rsa' exercises.tgz z4XY@login.hpc.cam.ac.uk:}}\hfill\\
The argument after the -e option tells the ssh used by rsync to employ the same private key we used in the last exercise.}
\end{description}}
\item{Switch back to the SSH session you created in the previous exercise. Verify that the file is now present by using \alert{ls}.}
\visible<3->{\begin{description}
\item[\emph{Hints:}]{Do \alert{ls -al exercise*}}
\end{description}}
\item{Unpack the tar archive to create an exercise subdirectory.}
\visible<4->{\begin{description}
\item[\emph{Hints:}]{Do \alert{tar -zxvf exercises.tgz}}
\end{description}}
\end{itemize}
\end{frame}

\section{Remote desktop}
\begin{frame}{Exercise 3: Remote desktop (OPTIONAL)}
\begin{itemize}
\item{Connect to the remote desktop running on \alert{login-gfx1.hpc.cam.ac.uk} on display \alert{99}. The VNC password is \alert{\emph{``trAin99''}}.}
\visible<2->{\begin{description}
\item[\emph{Hints:}]{\small Because the HPCS only allows SSH connections, to use VNC we need to tunnel via SSH. Since the MCS Linux systems do not have a VNC client application, we will use the java client. You will need to create a local port forwarding using ssh to both the VNC server port number \alert{$5900+99=5999$} and the java client web port number \alert{$5800+99=5899$} at localhost via \alert{login-gfx1.hpc.cam.ac.uk}:\hfill\\\smallskip
\rightline{\tiny\alert{ssh -i id\_rsa -L 5899:localhost:5899 -L 5999:localhost:5999 z4XY@login-gfx1.hpc.cam.ac.uk}}
Then point a web browser to \alert{http://localhost:5899}. You should be prompted to accept the TurboVNC java client and then for the VNC password which is \alert{\emph{``trAin99''}}. Note that this is a view-only password.}
\end{description}}
\end{itemize}
\end{frame}

\section{Modules and Compilers}
\begin{frame}{Exercise 4: Modules and Compilers}
\begin{itemize}
\item{Go to the \alert{exercises} directory of your HPCS training account.}
\visible<2->{\begin{description}
\item[\emph{Hints:}]{\small Firstly you may need to review Exercise~1 in order to reconnect to your HPCS training account. At the HPCS command prompt, change to the exercises directory (\alert{cd $\tilde{}$/exercises}).}
\end{description}}
\item{Try to compile the \alert{hello.c} program using the default \alert{icc} compiler (it will fail because there is a deliberate bug).}
\visible<3->{\begin{description}
\item[\emph{Hints:}]{\small \alert{icc hello.c -o hello}}
\end{description}}
\item{To fix the problem, open the \alert{hello.c} file in the \alert{gedit} editor.}
\visible<4->{\begin{description}
\item[\emph{Hints:}]{\small Launch gedit in the background by doing \alert{gedit\&}. A gedit window should appear. Remove the word \alert{BUG} and save the file.}
\end{description}}
\end{itemize}
\end{frame}

\begin{frame}{Exercise 4: Modules and Compilers (ctd)}
\begin{itemize}
\item{Try again to compile the \alert{hello.c} program using the default \alert{icc} compiler, and run it. You should see \alert{``\emph{node} says: Hello, World!''}.}
\visible<2->{\begin{description}
\item[\emph{Hints:}]{\small \alert{icc hello.c -o hello} then run: \alert{./hello}.}
\end{description}}
\item{Which version of \alert{icc} did you use? Find this out by listing the current modules loaded.}
\visible<3->{\begin{description}
\item[\emph{Hints:}]{\small \alert{module list} --- the Intel compiler modules are named \alert{intel/cce/\emph{version}}.  You can also work out the version directly by finding the location of the binary, e.g.\ doing\hfill\\
\alert{which icc} which should return the path:\hfill\\
\alert{/usr/local/Cluster-Apps/intel/cce/\emph{version}/bin/icc}.}
\end{description}}

\item{Compile and run the \alert{hello.c} program in the exercises directory using a non-default C compiler.}
\visible<4->{\begin{description}
\item[\emph{Hints:}]{\small E.g. load the latest PGI C compiler (\alert{pgcc}) with \alert{module load pgi}. module av will show all possible choices (not all of which are compilers).}
\end{description}}
\end{itemize}
\end{frame}

\section{Submitting Jobs}
\begin{frame}{Exercise 5: Submitting Jobs}
\begin{itemize}
\item{Submit a job which will run a copy of your hello program on all cores of the 4 12-core nodes which are available for training.}
\visible<2->{\begin{description}
\item[\emph{Hints:}]{\scriptsize\begin{enumerate}
\item{Edit the script \alert{job\_script} in your exercises directory. Set:\hfill\\
\alert{\#SBATCH --nodes=4}\hfill\\
\alert{\#SBATCH --ntasks=48}\hfill\\
\alert{application="./hello"}\hfill\\
In the module section, make sure that the module you used to compile \alert{hello} is also loaded (last).}
\item{Submit the job with \alert{sbatch job\_script}. The jobid is then printed.}
\item{Watch the job in the queue with \alert{squeue}.}
\item{After it has disappeared, open the output file \alert{slurm-jobid.out} in your editor. There should be 12 ``Hello, World!'' messages from 4 different nodes.}
\end{enumerate}%
}
\end{description}}
\end{itemize}
\end{frame}

\begin{frame}{Exercise 6: Submitting Jobs (ctd)}
\begin{itemize}
\item{Experiment with changing the number of nodes and tasks by changing and submitting job\_script (you are limited to 4 nodes in total).}
\end{itemize}
\end{frame}

\section{Array Jobs}
\begin{frame}{Exercise 7: Array Jobs}
\begin{itemize}
\item{Submit your last job in the form of an array with indices 2, 4 and 6. Use -H with sbatch to mark the array as held (so that it won't run immediately).}
\visible<2->{\begin{description}
\item[\emph{Hints:}]{\scriptsize\begin{enumerate}
\item{Use \alert{sbatch -H --array=2,4,6 job\_script}}
\item{Use \alert{squeue -u userid} to see your array job. Note that \alert{-r} reports each array element individually.}
\end{enumerate}%
}
\end{description}}
\item{Release array element 4 and allow it to run. Then release the other two.}
\visible<3->{\begin{description}
\item[\emph{Hints:}]{\scriptsize\begin{enumerate}
\item{Use \alert{scontrol release \$\{{\color[rgb]{0,0.6,0}SLURM\_ARRAY\_JOB\_ID}\}\_{{\color{red}4}}}}
\item{Use \alert{squeue -u userid} again to watch what happens.}
\item{When array element 4 has completed, release the others with\hfill\break
\null\qquad scontrol release \$\{{\color[rgb]{0,0.6,0}SLURM\_ARRAY\_JOB\_ID}\}\hfill\break
i.e. use the array id to release the entire array.}
\end{enumerate}%
}
\end{description}}
\end{itemize}
\end{frame}

\end{document}
