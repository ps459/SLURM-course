\part{Excercises}
\begin{frame}
\partpage
\end{frame}

\subsection{Exercise ...: Install the R library locally}
\begin{frame}[fragile]{Exercise ...: Install the R library locally}
As a user you can create a local R library directory for packages that you want to install. 

\begin{itemize}
\item Load an R module: 
module load r-3.4.3-gcc-5.4.0-rbvhnga
\item Create a folder in your home for your own R package installs:
mkdir ~/my-R-libs
\item Make R aware of the new library location:
\begin{verbatim}
echo "R_LIBS_USER=~/my-R-libs" \textgreater .Renviron
\end{verbatim}
\item Start R:
R
\item Display your library paths:
.libPaths()
\item Try loading a library:
require(pander)
\item Its not insalled, lets install it:
install.packages("pander")
\item Try loading a library:
require(pander)
\item Library is now installed, lets quit R:
quit()
\end{itemize}
\end{frame}

\subsection{Excercise explained}
\begin{frame}[fragile]{Excercise explained}
As a user you can create a local R library directory for packages that you want to install.
\begin{itemize}
\item First we load an R module, note the lower case r: module load r/(version). The lower case r is neccasary as upper case R modules are compiled for an older version of linux.
\item We create a folder for your local R package installs. This could be in your home or on HPC work.
mkdir ~/my-R-libs
\item In the root of your home directory we create a file .Renviron. This file is read by R and sets a variable which makes it aware of our local R library directory.
\item echo " " outputs the text between the quotes, > redirects the text into .Renviron which a file.
\item When we start R the .Renviron file is read and R will look for and install local packages in the local library directory.
\item .libPaths() is used to displays the R library locations where R modules will get installed. 
\item We then installing a package to the library using the R console:
install.packages("new_package_installation", "options", dependences=)
Note that its possible to specify options or name other packages that are a dependency. 
\item Now you can load the library from the R terminal or inside an Rscript:
require(pander)
\end{itemize}
\end{frame}
