% $Header: /Users/joseph/Documents/LaTeX/beamer/solutions/generic-talks/generic-ornate-15min-45min.en.tex,v 90e850259b8b 2007/01/28 20:48:30 tantau $

\documentclass{beamer}
%\documentclass[handout]{beamer}
\usepackage{pgfpages}
%\pgfpagesuselayout{4 on 1}[a4paper,border shrink=5mm,landscape]

% Based on solutions/generic-ornate-15min-45min.en.tex
% and HPCS-UoC-Beamer-master/uoc-beamer-template-1.tex.

% Copyright 2004 by Till Tantau <tantau@users.sourceforge.net>.
%
% In principle, this file can be redistributed and/or modified under
% the terms of the GNU Public License, version 2.
%
% However, this file is supposed to be a template to be modified
% for your own needs. For this reason, if you use this file as a
% template and not specifically distribute it as part of a another
% package/program, I grant the extra permission to freely copy and
% modify this file as you see fit and even to delete this copyright
% notice. 


\mode<presentation>
{
  \usetheme{cambridge}
  %\usetheme{Warsaw}

  \setbeamertemplate{navigation symbols}{}
  \setbeamercovered{transparent}
  % or whatever (possibly just delete it)
}

\usepackage[english]{babel}
\usepackage[latin1]{inputenc}
\usepackage{multicol}

%\usepackage{times}
%\usepackage[T1]{fontenc}
% Or whatever. Note that the encoding and the font should match. If T1
% does not look nice, try deleting the line with the fontenc.


\title[An introduction to HPC on Minerva --- Exercises] % (optional, use only with long paper titles)
{An Introduction to High Performance Computing on the Minerva Cluster --- Exercises}

%\subtitle
%{Presentation Subtitle} % (optional)

\author[SJ Rankin] % (optional, use only with lots of authors)
{Stuart Rankin\\ \texttt{sjr20@cam.ac.uk}}
%{F.~Author\inst{1} \and S.~Another\inst{2}}
% - Use the \inst{?} command only if the authors have different
%   affiliation.

\institute[UIS, University of Cambridge] % (optional, but mostly needed)
{Research Computing Services (http://www.hpc.cam.ac.uk/)\\
University Information Services (http://www.uis.cam.ac.uk/)}
% - Use the \inst command only if there are several affiliations.
% - Keep it simple, no one is interested in your street address.

\date[24/04/2018] % (optional)
{24th April 2018 / NPL User Training}

\subject{Courses --- Exercises}
% This is only inserted into the PDF information catalog. Can be left
% out. 



% If you have a file called "university-logo-filename.xxx", where xxx
% is a graphic format that can be processed by latex or pdflatex,
% resp., then you can add a logo as follows:

% \pgfdeclareimage[height=0.5cm]{university-logo}{university-logo-filename}
% \logo{\pgfuseimage{university-logo}}



% Delete this, if you do not want the table of contents to pop up at
% the beginning of each subsection:
%\AtBeginSubsection[]
%{
%  \begin{frame}<beamer>{Outline}
%    \tableofcontents[currentsection,currentsubsection]
%  \end{frame}
%}


% If you wish to uncover everything in a step-wise fashion, uncomment
% the following command: 

%\beamerdefaultoverlayspecification{<+->}


\begin{document}

\begin{frame}
  \titlepage
\end{frame}

\section{Login}
\begin{frame}{Exercise 1: Login}
\begin{itemize}
\item{Using MobaXterm, login to your Minerva account.}
\begin{description}
\item[\emph{Hints:}]{\small Start MobaXterm. Press \alert{Session} (top left) and \alert{SSH} in the settings panel which appears.\\\smallskip
The remote host is \alert{minerva-login1.npl.co.uk}. Specify the username --- this will be your AD identifier and of the form \alert{npl$\backslash$abXY}.\hfill\\\smallskip
\smallskip
The password will be your usual password in the NPL Active Directory (AD).}
\end{description}
\end{itemize}
\end{frame}


\section{Simple command line operations}
\begin{frame}{Exercise 2: Simple command line operations (i)}
\begin{itemize}

\item[(a)]{List your current directory (folder) using \alert{ls}. This won't show everything --- use \alert{ls -al} for a long listing showing all files. Initially you will start in your home directory --- use \alert{pwd} to print the name of your current working directory. If you get lost, you can always do \alert{cd} without arguments to return to your home directory.}

\item[(b)]{Focus your long listing on \alert{all files with names beginning ``exercise''}.}
 \visible<2->{\begin{description}
\item[\emph{Hints:}]{Do \alert{ls -al exercise*}}
\end{description}}

\item[(c)]{Print a long listing of the subdirectory \alert{hpc-work}.}
 \visible<3->{\begin{description}
\item[\emph{Hints:}]{Do \alert{ls -al hpc-work/}. Note that omitting the / reveals that the item hpc-work is actually a shortcut (technically a symbolic link) to \alert{/hpc-work/username}.}
 \end{description}}

\end{itemize}
\end{frame}

\begin{frame}{Exercise 2: Simple command line operations (ii)}
  \begin{itemize}
    
\item[(d)]{View the man page for the \alert{cp} command by doing \alert{man cp}. Use \alert{SPACE} to page down and \alert{b} to page up. Press \alert{q} to exit the manual page command.}

\item[(d)]{Copy \alert{exercises.tgz} to the $\tilde{}$/hpc-work directory. Note that $\tilde{}$ is just a convenient shorthand for your home directory. Omitting the $\tilde{}$/ will look for a hpc-work in the current directory.}
 \visible<2->{\begin{description}
\item[\emph{Hints:}]{Do \alert{cp exercises.tgz\quad$\tilde{}$/hpc-work/}. Note that you can often reduce typing by pressing \alert{TAB}.}
 \end{description}}

\item[(e)]{Use the \alert{cd} command to enter the $\tilde{}$/hpc-work directory and then list the contents --- you should see the copy of exercises.tgz.}
 \visible<3->{\begin{description}
\item[\emph{Hints:}]{Do \alert{cd\quad$\tilde{}$/hpc-work/} then \alert{ls -al}. Note that \alert{cd ..} will take you back up one step to the home directory.}
\end{description}}

\item{Unpack the tar archive to create an exercise subdirectory.}
\visible<4->{\begin{description}
\item[\emph{Hints:}]{Do \alert{tar -zxvf exercises.tgz}}
\end{description}}
\end{itemize}
\end{frame}

\section{File transfer}
\begin{frame}{Exercise 3: File transfer}
\begin{itemize}
\item{Using MobaXterm, SFTP the file \alert{exercises.tgz} from Minerva back to your local filespace.}
\visible<2->{\begin{description}
\item[\emph{Hints:}]{\small Start a SFTP session, using the same remote host and username as in the previous exercise.\\\smallskip
Drag the \alert{exercises.tgz} file from the remote Minerva folder (this will be your home directory on Minerva) to the local PC.}
\end{description}}
\end{itemize}
\end{frame}

\section{Remote desktop}
\begin{frame}{Exercise 4: Remote desktop}
  \begin{itemize}
    \item{Connect to Minerva and launch a remote desktop. You will need to set a (different!) password the first time. Note your unique display number.}
\item{Using MobaXterm, connect to the remote desktop running on \alert{minerva-login1.npl.co.uk} on the correct display number.}
\visible<2->{\begin{description}
\item[\emph{Hints:}]{\small Because the cluster only allows SSH connections from outside, to use VNC we need to tunnel via SSH.\hfill\\\smallskip
Use \alert{localhost} as the remote hostname, and set the Port to \alert{$5900+displaynumber$} (the reason for this is ancient history).\hfill\\\smallskip
Now go to \alert{Advanced VNC settings}, tick \alert{Connect through SSH gateway} and enter \alert{minerva-login1.npl.co.uk} as the gateway server, with your AD identifier \alert{npl$\backslash$abc12)} as the user. Click OK.\hfill\\\smallskip
You should be prompted first for your AD password, then for the VNC password.}
\end{description}}
\end{itemize}
\end{frame}

\section{Modules and Compilers}
\begin{frame}{Exercise 5: Modules and Compilers}
\begin{itemize}
\item{Go to the \alert{exercises} directory of your Minerva account.}
\visible<2->{\begin{description}
\item[\emph{Hints:}]{\small Firstly you may need to review Exercise~1 in order to reconnect to your Minerva account. (Note that your earlier SSH session may in fact be saved on the left side of the MobaXterm GUI.) Alternatively, use your VNC desktop session. At the Minerva command prompt, change to the exercises directory (\alert{cd $\tilde{}$/hpc-work/exercises}).}
\end{description}}
\item{Try to compile the \alert{hello.c} program using the default \alert{gcc} compiler (it will fail because there is a deliberate bug).}
\visible<3->{\begin{description}
\item[\emph{Hints:}]{\small \alert{gcc hello.c -o hello}}
\end{description}}
\item{To fix the problem, open the \alert{hello.c} file in the \alert{gedit} editor.}
\visible<4->{\begin{description}
\item[\emph{Hints:}]{\small Launch gedit in the background by doing \alert{gedit\&}. A gedit window should appear. Remove the word \alert{BUG}, save the file and recompile. Do \alert{./hello} to run the program.}
\end{description}}
\end{itemize}
\end{frame}

\begin{frame}{Exercise 5: Modules and Compilers (ctd)}
  \begin{itemize}
  \item{The default version of gcc is 4.8.5. Compile hello.c again with \alert{gcc 5.4.0}.}
   \visible<2->{\begin{description}
\item[\emph{Hints:}]{\small module av, module load, then \alert{gcc hello.c -o hello2}}
\end{description}}
\item{Launch the Matlab GUI. Note this should work from either the SSH command-line or remote desktop sessions.}
\visible<3->{\begin{description}
\item[\emph{Hints:}]{\small \alert{module load matlab} then run: \alert{matlab\&}}
\end{description}}
\item{Quit Matlab and launch it again without the graphical desktop interface. This is the way to launch it inside a batch job.}
\visible<4->{\begin{description}
\item[\emph{Hints:}]{\alert{matlab -nodisplay -nojvm -nosplash}}
\end{description}}
\item{Launch the COMSOL GUI.}
\visible<5->{\begin{description}
\item[\emph{Hints:}]{Search for and load the module, then run \alert{comsol}.}
\end{description}}
\end{itemize}
\end{frame}

\section{Submitting Jobs}
\begin{frame}{Exercise 6: Submitting Jobs}
\begin{itemize}
\item{Submit a job which will run \alert{matlab} on the \alert{file.m} command file (which contains just the \alert{ver} command).}
\visible<2->{\begin{description}
\item[\emph{Hints:}]{\scriptsize\begin{enumerate}
\item{Load the matlab module at the place indicated in the file \alert{job\_script} in your exercises directory.}
\item{Set the value of application to\hfill\break{}\alert{\"{}matlab -nodesktop -nosplash -nojvm\"{}}}
\item{Set the value of options to \alert{\"{}-r file\"{}}}
\item{Submit the job with \alert{sbatch job\_script}. The jobid is then printed.}
\item{Watch the job in the queue with \alert{squeue}.}
\item{After it has disappeared, open the output file \alert{slurm-jobid.out} in your editor. It should contain a list of licensed Matlab features.}
  \item{For more demanding work you can increase the available memory by increasing the number of cpus.}
  \end{enumerate}%
}
\end{description}}
\end{itemize}
\end{frame}

\begin{frame}{Exercise 6: Submitting Jobs (ctd)}
\begin{itemize}
\item{Submit a job which will run a copy of your hello program on 1 cpu.}
\visible<2->{\begin{description}
\item[\emph{Hints:}]{\scriptsize\begin{enumerate}
\item{Edit the script \alert{job\_script} in your exercises directory. Set:\hfill\\
\alert{\#SBATCH --nodes=1}\hfill\\
\alert{\#SBATCH --ntasks=1}\hfill\\
\alert{application="./hello"}}
\item{Submit the job with \alert{sbatch job\_script}. The jobid is then printed.}
\item{Watch the job in the queue with \alert{squeue}.}
\item{After it has disappeared, open the output file \alert{slurm-jobid.out} in your editor. There should be exactly one ``Hello, World!'' message.}
\end{enumerate}%
}
\end{description}}
\end{itemize}
\end{frame}

\section{Array Jobs}
\begin{frame}{Exercise 7: Array Jobs}
\begin{itemize}
\item{Submit your last job in the form of an array with indices 1-64. Use -H with sbatch to mark the array as held (so that it won't run immediately).}
\visible<2->{\begin{description}
\item[\emph{Hints:}]{\scriptsize\begin{enumerate}
\item{Use \alert{sbatch -H -{}-array=1-64 job\_script}}
\item{Use \alert{squeue -u userid} to see your array job. Note that \alert{-r} reports each array element individually.}
\end{enumerate}%
}
\end{description}}
\item{Release array element 1 and allow it to run. Then release the others.}
\visible<3->{\begin{description}
\item[\emph{Hints:}]{\scriptsize\begin{enumerate}
\item{Use \alert{scontrol release \$\{{SLURM\_ARRAY\_JOB\_ID}\}\_{{\color{red}1}}}}
\item{Use \alert{squeue -u userid} again to watch what happens.}
\item{Release the others with\hfill\break
\null\qquad scontrol release \$\{{SLURM\_ARRAY\_JOB\_ID}\}\hfill\break
i.e. use the array id to release the entire array.}
  \item{When all the jobs complete you should have 64 slurm-\$\{SLURM\_ARRAY\_JOB\_ID\}\_N.out files saying hello from various cpus on possibly multiple nodes.}
\end{enumerate}%
}
\end{description}}
\end{itemize}
\end{frame}

\end{document}
