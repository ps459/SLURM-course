% $Header: /Users/joseph/Documents/LaTeX/beamer/solutions/generic-talks/generic-ornate-15min-45min.en.tex,v 90e850259b8b 2007/01/28 20:48:30 tantau $

\documentclass{beamer}
%\documentclass[handout]{beamer}
\usepackage{pgfpages}
%\pgfpagesuselayout{4 on 1}[a4paper,border shrink=5mm,landscape]

% Based on solutions/generic-ornate-15min-45min.en.tex
% and HPCS-UoC-Beamer-master/uoc-beamer-template-1.tex.

% Copyright 2004 by Till Tantau <tantau@users.sourceforge.net>.
%
% In principle, this file can be redistributed and/or modified under
% the terms of the GNU Public License, version 2.
%
% However, this file is supposed to be a template to be modified
% for your own needs. For this reason, if you use this file as a
% template and not specifically distribute it as part of a another
% package/program, I grant the extra permission to freely copy and
% modify this file as you see fit and even to delete this copyright
% notice. 


\mode<presentation>
{
  \usetheme{cambridge}
  %\usetheme{Warsaw}

  \setbeamertemplate{navigation symbols}{}
  \setbeamercovered{transparent}
  % or whatever (possibly just delete it)
}

\usepackage[english]{babel}
\usepackage[latin1]{inputenc}
\usepackage{multicol}

%\usepackage{times}
%\usepackage[T1]{fontenc}
% Or whatever. Note that the encoding and the font should match. If T1
% does not look nice, try deleting the line with the fontenc.


\title[An Introduction to HPC --- Exercises] % (optional, use only with long paper titles)
{An Introduction to HPC --- Exercises}

%\subtitle
%{Presentation Subtitle} % (optional)

\author[SJ Rankin] % (optional, use only with lots of authors)
{Stuart Rankin\\ \texttt{sjr20@cam.ac.uk}}
%{F.~Author\inst{1} \and S.~Another\inst{2}}
% - Use the \inst{?} command only if the authors have different
%   affiliation.

\institute[UIS, University of Cambridge] % (optional, but mostly needed)
{Research Computing Services (http://www.hpc.cam.ac.uk/)\\
University Information Services (http://www.uis.cam.ac.uk/)}
% - Use the \inst command only if there are several affiliations.
% - Keep it simple, no one is interested in your street address.

\date[31/07/2019] % (optional)
{31st July 2019 / UIS Training}

\subject{Courses --- Exercises}
% This is only inserted into the PDF information catalog. Can be left
% out. 



% If you have a file called "university-logo-filename.xxx", where xxx
% is a graphic format that can be processed by latex or pdflatex,
% resp., then you can add a logo as follows:

% \pgfdeclareimage[height=0.5cm]{university-logo}{university-logo-filename}
% \logo{\pgfuseimage{university-logo}}



% Delete this, if you do not want the table of contents to pop up at
% the beginning of each subsection:
%\AtBeginSubsection[]
%{
%  \begin{frame}<beamer>{Outline}
%    \tableofcontents[currentsection,currentsubsection]
%  \end{frame}
%}


% If you wish to uncover everything in a step-wise fashion, uncomment
% the following command: 

%\beamerdefaultoverlayspecification{<+->}


\begin{document}

\begin{frame}
  \titlepage
\end{frame}

\section{Login}
\begin{frame}{Exercise 1: Login}
\begin{itemize}
\item{Log into your RCS training account.}
\begin{description}
\item[\emph{Hints:}]{\small Create a terminal window and use \alert{ssh} to login to your cluster training account. \\\smallskip
The remote host is \alert{login.hpc.cam.ac.uk}. The user name is the same name as your MCS Desktop training account (i.e. \alert{z4XY}).\hfill\\\smallskip
{\scriptsize N.B.\ If in doubt about the name of your training account, check the number of your station (see the label on the top of the box), then station 1\textbf{XY} should correspond to account z4\textbf {XY}.}\hfill\\\smallskip
}
\end{description}
\end{itemize}
\end{frame}


\section{Simple command line operations}
\begin{frame}{Exercise 2: Simple command line operations}
\begin{itemize}

\item[(a)]{List your current directory (folder) using \alert{ls -al}. Use \alert{df -h} to see the various cluster filesystems, their sizes and their current total usages. You will be on a random login node -- use \alert{hostname} to confirm which one, and \alert{w} to find out who else is using it. Use \alert{lstopo} to find out more about the internal structure of the login node.}

\item[(b)]{Examine your personal filesystem quotas with the command \alert{quota}.}
 \visible<2->{\begin{description}
   \item{You should see a 40GB quota on /home, a 1TB block and 1024k file quota on /rds-d2 (which corresponds to $\tilde{}$/rds/hpc-work).}
   \end{description}}
 
\item[(c)]{Ask the scheduler what compute resources are available to you with \alert{mybalance}. This command may take a little while to return (the units are CPU/GPU/KNL hours).}

\end{itemize}
\end{frame}

\section{File transfer}
\begin{frame}{Exercise 3: File transfer}
\begin{itemize}
\item{Use SFTP to transfer the file \alert{exercises.tgz} to your Research Computing Service training account directory $\tilde{}$/rds/hpc-work.}
\visible<2->{\begin{description}
\item[\emph{Hints:}]{\small The command is \alert{sftp}. Use the same remote host, username and password as in the previous exercise.\\\smallskip
  Use \alert{cd rds/hpc-work} to change the target directory, then \alert{put exercises.tgz} to transfer the file from your MCS home directory to the target directory on the Research Computing Service cluster. Use \alert{quit} to close the connection.}
\end{description}Optionally, copy the file over again using \alert{rsync}.}
\end{itemize}
\end{frame}
    
\section{File transfer}
\begin{frame}{Exercise 3: File transfer (ctd)}
\begin{itemize}
\item{Switch back to the SSH session you created in the previous exercise. Verify that the file is now present by using \alert{ls}.}
\visible<2->{\begin{description}
\item[\emph{Hints:}]{Do \alert{ls -al\quad$\tilde{}$/rds/hpc-work/}. Note that you can often reduce typing by pressing \alert{TAB}.}
\end{description}}
\item{Unpack the tar archive to create an exercise subdirectory.}
\visible<3->{\begin{description}
\item[\emph{Hints:}]{Do \alert{cd\quad$\tilde{}$/rds/hpc-work/} then \alert{tar -zxvf exercises.tgz}.}
\end{description}}
\end{itemize}
\end{frame}
    

%\section{Remote desktop}
%\begin{frame}{Exercise 4: Remote desktop}
%\begin{itemize}
%\item{Using TurboVNC, connect to the remote desktop running on \alert{login-gfx2.hpc.cam.ac.uk} %on display \alert{:99}. The command is \alert{\~{}/bin/turbovncviewer} (note that the vncviewer %command provides a different version) and the VNC password is \alert{\emph{``trAin99''}}.}
%\visible<2->{\begin{description}
%\item[\emph{Hints:}]{\small Because the RCS clusters only allow SSH connections, to use VNC we %need to tunnel via SSH.\hfill\\\smallskip
%Use \alert{localhost:99} as the remote display name.\hfill\\\smallskip
%Use \alert{-via USER@login-gfx2.hpc.cam.ac.uk} to specify the gateway server, with your training account ID as USER.\hfill\\\smallskip
%You should be prompted first for your training account password, then for the VNC password which is \alert{\emph{``trAin99''}}. Note that this is a view-only password.}
%\end{description}}
%\end{itemize}
%\end{frame}


\section{Modules and Compilers}
\begin{frame}{Exercise 5: Modules and Compilers}
\begin{itemize}
\item{Go to the \alert{exercises} directory of your cluster account.}
\visible<2->{\begin{description}
\item[\emph{Hints:}]{\small Firstly you may need to review Exercise~1 in order to reconnect to your cluster account. At the remote command prompt, change to the exercises directory (\alert{cd $\tilde{}$/rds/hpc-work/exercises}).}
\end{description}}
\item{Try to compile the \alert{hello.c} program using the default \alert{gcc} compiler (it will fail because there is a deliberate bug).}
\visible<3->{\begin{description}
\item[\emph{Hints:}]{\small \alert{gcc hello.c -o hello}}
\end{description}}
\item{To fix the problem, open the \alert{hello.c} file in an editor (e.g. \alert{gedit}, \alert{nano}, \alert{emacs}).}
\visible<4->{\begin{description}
\item[\emph{Hints:}]{\small Launch gedit in the background by doing \alert{gedit\&}. A gedit window should appear. Remove the word \alert{BUG}, save the file and recompile. Do \alert{./hello} to run the program.}
\end{description}}
\end{itemize}
\end{frame}

\begin{frame}{Exercise 5: Modules and Compilers (ctd)}
  \begin{itemize}
  \item{The default version of gcc on the RCS HPC clusters is 4.8.5. Compile hello.c again with \alert{gcc 5.4.0}.}
   \visible<2->{\begin{description}
\item[\emph{Hints:}]{\small module av, module load, then \alert{gcc hello.c -o hello2}}
\end{description}}
\item{Launch the Matlab GUI. Note this should work from either the SSH command-line or remote desktop sessions.}
\visible<3->{\begin{description}
\item[\emph{Hints:}]{\small \alert{module load matlab} then run: \alert{matlab\&}}
\end{description}}
\item{Quit Matlab and launch it again without the graphical desktop interface. This is the way to launch it inside a batch job.}
\visible<4->{\begin{description}
\item[\emph{Hints:}]{\alert{matlab -nodisplay -nojvm -nosplash}}
\end{description}}
\end{itemize}
\end{frame}

\section{Submitting Jobs}
\begin{frame}{Exercise 6: Submitting Jobs (Matlab)}
\begin{itemize}
\item{Submit a job which will run \alert{matlab} on the \alert{file.m} command file (which contains just the Matlab \alert{ver} command).}
\visible<2->{\begin{description}
\item[\emph{Hints:}]{\scriptsize\begin{enumerate}
\item{Load the matlab module at the place indicated in the file \alert{job\_script} in your exercises directory.}
\item{Set the value of application to\hfill\break{}\alert{\"{}matlab -nodesktop -nosplash -nojvm\"{}}}
\item{Set the value of options to \alert{\"{}-r file\"{}}}
\item{Submit the job with \alert{sbatch job\_script}. The jobid is then printed.}
\item{Watch the job in the queue with \alert{squeue}.}
\item{After it has disappeared, open the output file \alert{slurm-jobid.out} in your editor. It should contain a list of licensed Matlab features from the ver command.}
  \item{For more demanding work you can increase the available memory by increasing the number of cpus.}
  \end{enumerate}%
}
\end{description}}
\end{itemize}
\end{frame}

\begin{frame}{Exercise 7: Submitting Jobs (serial or threaded application)}
\begin{itemize}
\item{Submit a job which will run a copy of your hello program on 1 cpu.}
\visible<2->{\begin{description}
\item[\emph{Hints:}]{\scriptsize\begin{enumerate}
\item{Edit the script \alert{job\_script} in your exercises directory. Set:\hfill\\
\alert{\#SBATCH --nodes=1}\hfill\\
\alert{\#SBATCH --ntasks=1}\hfill\\
\alert{application="./hello"}}
\item{Submit the job with \alert{sbatch job\_script}. The jobid is then printed.}
\item{Watch the job in the queue with \alert{squeue}.}
\item{After it has disappeared, open the output file \alert{slurm-jobid.out} in your editor. There should be exactly one ``Hello, World!'' message.}
\end{enumerate}%
}
\end{description}}
Experiment with varying the number of nodes and tasks (you are limited to 4 nodes). Note you will need to launch the application with \alert{srun} to actually use more than 1 cpu.
\end{itemize}
\end{frame}

\begin{frame}{Exercise 8: Submitting Jobs (R)}
\begin{itemize}
\item{R jobs may be serial, threaded, or even MPI parallel depending on the packages used. Submit a job which will run the trivial script \alert{hello.r} program on 1 cpu.}
\visible<2->{\begin{description}
\item[\emph{Hints:}]{\scriptsize\begin{enumerate}
\item{Edit the script \alert{job\_script} in your exercises directory. Set:\hfill\\
\alert{\#SBATCH --nodes=1}\hfill\\
\alert{\#SBATCH --ntasks=1}\hfill\\
\alert{application="Rscript"}\hfill\\
\alert{options="hello.r"}}
\item{Submit the job with \alert{sbatch job\_script}. The jobid is then printed.}
\end{enumerate}%
}
\end{description}}
\item{Repeat this using a different version of R.}
\end{itemize}
\end{frame}

\section{Array Jobs}
\begin{frame}{Exercise 9: Array Jobs}
\begin{itemize}
\item{Submit your last job in the form of an array with indices 1-64. Use -H with sbatch to mark the array as held (so that it won't run immediately).}
\visible<2->{\begin{description}
\item[\emph{Hints:}]{\scriptsize\begin{enumerate}
\item{Use \alert{sbatch -H -{}-array=1-64 job\_script}}
\item{Use \alert{squeue -u userid} to see your array job. Note that \alert{-r} reports each array element individually.}
\end{enumerate}%
}
\end{description}}
\item{Release array element 1 and allow it to run. Then release the others.}
\visible<3->{\begin{description}
\item[\emph{Hints:}]{\scriptsize\begin{enumerate}
\item{Use \alert{scontrol release \$\{{SLURM\_ARRAY\_JOB\_ID}\}\_{{\color{red}1}}}}
\item{Use \alert{squeue -u userid} again to watch what happens.}
\item{Release the others with\hfill\break
\null\qquad scontrol release \$\{{SLURM\_ARRAY\_JOB\_ID}\}\hfill\break
i.e. use the array id to release the entire array.}
  \item{When all the jobs complete you should have 64 slurm-\$\{SLURM\_ARRAY\_JOB\_ID\}\_N.out files saying hello from various cpus on possibly multiple nodes.}
\end{enumerate}%
}
\end{description}}
\end{itemize}
\end{frame}

\end{document}
